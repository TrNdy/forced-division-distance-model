\documentclass[a4paper]{article}
\usepackage[T1]{fontenc}
\usepackage{amssymb}
\usepackage{mathtools}
%\usepackage{amsmath}
\usepackage{amsfonts}
\usepackage{amsthm}
\usepackage{xspace}
\usepackage[usenames,dvipsnames]{color}




%%%% macros %%%%%%%%%%%%%%%%%%%%%%%%%%%%%%%%%%%%%%%%%%%%%%%%%%%%%%%%%%%%%%%%%%%%%%%%%%%%

\newcommand{\FJ}[1]{{\color{Cyan}\sc \textbf{FJ:} #1}{}}
\newcommand{\TP}[1]{{\color{Red}\sc \textbf{TP:} #1}{}}


 
\newcommand{\set}[1]{\ensuremath{\{#1\}}}

% the set of integers (Z)
\newcommand{\Int}{\ensuremath{\mathbb{Z}}}


% segmentation hypotheses  - - - - - - - - - - - - - - - - - - - - - - - - - - 
\newcommand{\hypset}[1]{\ensuremath{H^{(#1)}}\xspace}
\newcommand{\Hypt}{\hypset{t}}

\newcommand{\hyp}[1][]{\ensuremath{h_{#1}}\xspace}
\newcommand{\hypt}[1][t]{\ensuremath{h^{(#1)}}\xspace}
\newcommand{\hypti}[2][t]{\ensuremath{h^{(#1)}_{#2}}\xspace}

% segmentation variables - - - - - - - - - - - - - - - - - - - - - - - - - - - 
\newcommand{\vhset}[1]{\hypset{#1}}
\newcommand{\Ht}[1][t]{\vhset{#1}}

\newcommand{\vh}[1][]{\ensuremath{h_{#1}}\xspace}
\newcommand{\vht}[1][t]{\ensuremath{h^{(#1)}}\xspace}
\newcommand{\vhtd}[2][t]{\ensuremath{h^{(#1)}\left({#2}\right)}\xspace}
\newcommand{\vhti}[2][t]{\ensuremath{h^{(#1)}_{#2}}\xspace}
\newcommand{\vhtid}[3][t]{\ensuremath{h^{(#1)}_{#2}\left({#3}\right)}\xspace}

% set of root-to-leaf paths
\newcommand{\Paths}[1]{\ensuremath{\mathcal{P}(#1)}\xspace}
\newcommand{\Path}{\ensuremath{\pi}\xspace}

% assignment variables - - - - - - - - - - - - - - - - - - - - - - - - - - - -
\newcommand{\vaset}[1]{\ensuremath{A^{(#1)}}\xspace}
\newcommand{\At}[1][t]{\vaset{#1}}

\newcommand{\va}[1][]{\ensuremath{a_{#1}}\xspace}
\newcommand{\vat}[1][t]{\ensuremath{a^{(#1)}}\xspace}
\newcommand{\vati}[2][t]{\ensuremath{a^{(#1)}_{#2}}\xspace}

% mapping
\newcommand{\vamset}[1]{\ensuremath{A^{(#1)}_{\mapsto}}\xspace}
\newcommand{\Amt}[1][t]{\vamset{#1}}
\newcommand{\vam}[3]{\ensuremath{a^{(#1)}_{#2\mapsto#3}}\xspace}
\newcommand{\vamd}[4]{\ensuremath{a^{(#1)}_{#2\mapsto#3}\left(#4\right)}\xspace}

% division
\newcommand{\vadset}[1]{\ensuremath{A^{(#1)}_{\div}}\xspace}
\newcommand{\Adt}[1][t]{\vadset{#1}}
\newcommand{\vad}[3]{\ensuremath{a^{(#1)}_{#2\div#3}}\xspace}

% appearance
\newcommand{\vaaset}[1]{\ensuremath{A^{(#1)}_{\div}}\xspace}
\newcommand{\Aat}[1][t]{\vaaset{#1}}
\newcommand{\vaa}[2]{\ensuremath{a^{(#1)}_{\nearrow#2}}\xspace}

% disappearance (exit)
\newcommand{\vaeset}[1]{\ensuremath{A^{(#1)}_{\searrow}}\xspace}
\newcommand{\Aet}[1][t]{\vaeset{#1}}
\newcommand{\vae}[2]{\ensuremath{a^{(#1)}_{#2\searrow}}\xspace}
\newcommand{\vaed}[3]{\ensuremath{a^{(#1)}_{#2\searrow}\left(#3\right)}\xspace}

% assignment sets
\newcommand{\Gr}[1]{\ensuremath{\Gamma_{\text{R}}\left(#1\right)}\xspace}
\newcommand{\Gl}[1]{\ensuremath{\Gamma_{\text{L}}\left(#1\right)}\xspace}
\newcommand{\Aup}[1]{\ensuremath{A_\uparrow(#1)}\xspace}

% factor graph - - - - - - - - - - - - - - - - - - - - - - - - - - - - - - - 
\newcommand{\FGV}{\ensuremath{\mathcal{V}}\xspace}
\newcommand{\FGVH}{\ensuremath{\mathcal{H}}\xspace}
\newcommand{\FGVA}{\ensuremath{\mathcal{A}}\xspace}

\newcommand{\FGF}{\ensuremath{\mathcal{F}}\xspace}
\newcommand{\FGE}{\ensuremath{\mathcal{E}}\xspace}

% abbreviations  - - - - - - - - - - - - - - - - - - - - - - - - - - - - - - 
\newcommand{\Ie}{I.e.{}}
\newcommand{\Eg}{E.g.{}}
\newcommand{\Cf}{Cf.{}}
\newcommand{\ie}{i.e.{}}
\newcommand{\eg}{e.g.{}}
\newcommand{\cf}{cf.{}}
\newcommand{\etc}{etc\xspace}
\newcommand{\etal}{\emph{et~al.}}









\begin{document}

\title{Cell Tracking With Forced Division Distance}
\author{Tobias Pietzsch, Florian Jug}
\maketitle



We use a \emph{factor graph} $\mathcal{FG} = (\FGV, \FGF, \FGE)$ with
\FGV being a set of binary variables or \emph{variable nodes},
\FGF being a set of factors or \emph{factor nodes}, and
$\FGE \subset \FGV \times \FGF$~\cite{Frey:1997p3532}.
%
%
\section{Variable nodes.}
%
%
The variable nodes $\FGV = \FGVH\, \cup\, \FGVA$ comprise
\begin{itemize}
\item \emph{segmentation variables} $\FGVH = \bigcup_{t=1}^T \Ht$, and 
\item \emph{assignment variables} $\FGVA = \bigcup_{t=1}^{T-1} \At$.
\end{itemize}

%
\subsection{Segmentation variables}
%----------------------------------
Each binary segmentation variable $\vht \in \Ht$ indicates whether a particular segment hypothesis at time-point $t$ is choosen as part of the solution.\footnote{%
  To keep notation simple, we will use \vht to denote both, the binary segmentation variable and the corresponding segment hypothesis.
  Likewise, \Ht denotes the set of segmentation variables as well as the set of segment hypotheses, at time-point $t$.}

%
\subsection{Assignment variables}
%--------------------------------
Assignment variables $\vat \in \At$ link segment hypotheses at time-point $t$ to segment hypotheses at time-point $t+1$.
We distinguish three types of assignment variables.
%
\paragraph{Mapping assignments:}
A mapping assignment \vam{t}{i}{j} connects two segment hypotheses \vhti{i} and \vhti[t+1]{j}.
It indicates that these segments correspond to the same segmented cell that is tracked between time-points $t$ and $t+1$.
%
\paragraph{Division assignments:}
A division assignment \vad{t}{i}{jk} connects segment hypothesis \vhti{i} to \vhti[t+1]{j} and \vhti[t+1]{k}.
It indicates that these segments correspond to a cell division event, where one segmented cell at time-point $t$ divides into two daughter cells at $t+1$.
%
\paragraph{Appearance assignments:}
An appearance assignment \vaa{t}{i} is connected to one segment hypothesis \vhti[t+1]{i}. It indicates that this segment appears (enters the field of view) at time-point $(t+1)$. Note, that we annotate the assignment with time index $(t)$ although it concerns only hypotheses at $(t+1)$. This will be convenient later because all ``incoming'' assignments into $(t+1)$ have time index $(t)$.
%
\paragraph{Disappearance assignments:}
An disappearance (exit) assignment \vae{t}{i} is only connected to one segment hypothesis \vhti{i}. It indicates that this segment corresponds to a segmented cell at time-point $t$ that disappears (leaves the field of view) at time-point $t+1$.

%
\subsection{Division distance augmentation}
%------------------------------------------
\subsubsection{Segmentation variables}
%
To encode distance to last division, we replace each binary segmentation variable \vht by a vector of binary variables \vhtd{\delta} where $\delta\in\set{0,\dots,\Delta}$ denotes the number of time-points since the last division (clamped at $\Delta$).

We use \vht to denote the sum
\begin{equation}
	\vht = \sum_{0 \leq \delta \leq \Delta} \vhtd{\delta}.
\end{equation}
We add constraints to ensure that at most one \vhtd{\delta} can be active, effectively ensuring that $\vht$ can be treated as a binary variable.
\begin{equation}\label{eq:atmostonehdelta}
  \forall t \in \set{1, \dots, T},\;
  \forall \vht \in \Ht:
		\sum_{0 \leq \delta \leq \Delta} \vhtd{\delta}
  	\leq 1
\end{equation}
We take the liberty to write $\vht \in \Ht$ where it would be more accurate to write $\vht \subset \Ht$.


\subsubsection{Assignment variables}
%
\newcommand{\vamtij}{\vam{t}{i}{j}}
\newcommand{\vamtijd}[1][\delta]{\vamd{t}{i}{j}{#1}}
We replace each binary mapping variable \vamtij by a vector of binary variables \vamtijd where $\delta\in\set{0,\dots,\Delta}$ denotes the number of time-points since the last division (clamped at $\Delta$) at the start of the mapping.

Specifically, mapping assignment \vamtijd connects the two segment hypotheses \vhtid{i}{\delta} and \vhtid[t+1]{j}{\delta+1} (clamping at $\Delta$).

We use \vamtij to denote the sum
\begin{equation}
	\vamtij = \sum_{0 \leq \delta \leq \Delta} \vamtijd.
\end{equation}
We add constraints to ensure that at most one \vamtijd can be active, effectively ensuring that \vamtij can be treated as a binary variable.
\begin{equation}\label{eq:atmostoneamdelta}
  \forall t \in \set{1, \dots, T},\;
  \forall \vamtij \in \Amt:
		\sum_{0 \leq \delta \leq \Delta} \vamtijd
  	\leq 1
\end{equation}
%

\newcommand{\vaeti}{\vae{t}{i}}
\newcommand{\vaetid}[1][\delta]{\vaed{t}{i}{#1}}
\noindent
Similarly, we replace each disappearance assignment \vaeti by a vector of binary variables \vaetid where $\delta\in\set{0,\dots,\Delta}$ denotes the number of time-points since the last division (clamped at $\Delta$) of the exiting hypothesis \vhtid{i}{\delta}.

We use \vaeti to denote the sum
\begin{equation}
	\vaeti = \sum_{0 \leq \delta \leq \Delta} \vaetid.
\end{equation}
We add constraints to ensure that at most one \vaetid can be active, effectively ensuring that \vaeti can be treated as a binary variable.
\begin{equation}\label{eq:atmostoneaedelta}
  \forall t \in \set{1, \dots, T},\;
  \forall \vaeti \in \Amt:
		\sum_{0 \leq \delta \leq \Delta} \vaetid
  	\leq 1
\end{equation}
%
%
\TP{TODO: Constraints~\eqref{eq:atmostoneamdelta},~\eqref{eq:atmostoneaedelta} should be ensured by other constraints, so we do not have to add them explicitly. Find out whether and where this happens.}
%
%
%
\section{Factor nodes.}
%
%
Factor nodes connect to one or more variable nodes, assigning a potential to each joint configuration of these variables.
The factor nodes \FGF comprise unary factors and higher order factors.
Unary factors $f(v)$ are connected to each binary variable $v \in \FGV$, capturing the plausibility that $v$ is active given the data. Formally we define
\begin{equation}
  -\ln f(v) = \begin{cases}
    0    & \text{if } v = 0 \\
    c_v  & \text{if } v = 1,
  \end{cases}
\end{equation}
$c_{v}$ is the cost for including the respective segmentation or assignment variable in the solution.
These costs are derived from (image) features as described in the next subsection.

Structural constraints are expressed as $n$-ary factors
\begin{equation}
  -\ln f(var(E)) = \begin{cases}
    0    & \text{if $E$ holds} \\
    \infty & \text{otherwise,}
  \end{cases}
\end{equation}
where $E$ are (in-)equalities on the set of variables $var(E)$ connected to the factor.
The constraints formalized by these (in-)equalities prohibit
  solutions involving conflicting segmentation hypotheses or assignments that are inconsistent with the selected segmentation.\footnote{%
  Such inconsistent solutions correspond to events with infinite costs or $0$ probability.}
Constraints are described in Section~\ref{sec:constraints}.


%
%
\subsection{Costs.}
\label{sec:segcosts}
%
%
All costs $c_v$ corresponding to activating a variables $v$ are defined according to the following considerations.

\TP{TODO}

% = = = = = = = = = = = = = = = = = = = = = = = = = = = = = = = = = = = = = = = = = = =
\subsection{Constraints}
\label{sec:constraints}
% = = = = = = = = = = = = = = = = = = = = = = = = = = = = = = = = = = = = = = = = = = =
%
%
\subsubsection{Tree constraints.}
%
%
\TP{TODO: Revise for general conflict graph constraints. Singleton hypothesis constraints need to be added because they will ensure~\eqref{eq:atmostonehdelta}.}

We say that segment hypotheses $\hypti{i} \supset \hypti{j}$ are conflicting because they offer mutually exclusive interpretations of (parts of) the image data.
Tree constraints enforce that conflicting segment variables cannot be simultaneously active.
This is formalized in the set of inequalities
\begin{equation}\label{eq:treeconstraints}
  \forall t \in \set{1, \dots, T},\;
  \forall \Path \in \Paths{\Ht}:
  \sum_{\vht \in \Path} \vht \leq 1
\end{equation}
where \Paths{\Ht} is the set of all paths \Path from the root node in $(\Hypt,\supset)$ to any of its leaf nodes.
%
%
\subsubsection{Continuity constraints.}
%
%
Continuity constraints enforce consistency between segmentation and assignment variables.
If a segment hypothesis is selected,
   exactly one of the assignments entering it from the previous time-point,
   and exactly one of the assignments leaving it towards the next time-point must be selected as well.
If a segment hypothesis is not selected, neither must any of these assignments be selected.
This is formalized as the following sets of constraints.
For the entering assignments we have
\begin{equation}\label{eq:contenter}
  \forall\, 1 < t \le T,\;
  0 \le \delta \le \Delta,\;
  \vht \in \Ht:
    \sum_{\vat[t-1] \in \Gl{\vhtd{\delta}}} \vat[t-1]
  = \vhtd{\delta}
\end{equation}
where the \emph{left neighborhood} \Gl{\vh} is the set of all assignments entering \vh from the previous time-point.
That is,
\begin{equation}
	\Gl{\vhtid{i}{\delta}}
	= \begin{Bmatrix*}[l]
			\; \vamd{t-1}{j}{i}{\delta-1}            & \forall j \text{, if } \delta \ge 1, \\[4mm]
			\; \vad{t-1}{j}{ik},\; \vad{t-1}{j}{ki}  & \forall j, k \text{, if } \delta = 0,\; \\[4mm]
			\; \vaa{t-1}{i}                          & \text{if } \delta = \Delta, \\[4mm]
			\; \vamd{t-1}{j}{i}{\Delta}              & \text{if } \delta = \Delta
		\end{Bmatrix*}.
\end{equation}
%

\noindent
Similarly, for the assignments leaving to the next time-point we have
\begin{equation}\label{eq:contexit}
  \forall\, 1 \le t < T,\;
  0 \le \delta \le \Delta,\;
  \vht \in \Ht:
    \sum_{\vat \in \Gr{\vhtd{\delta}}} \vat
  = \vhtd{\delta}
\end{equation}
where the \emph{right neighborhood} \Gr{\vh} is the set of all assignments leaving \vh to the next time-point.
That is,
\begin{equation}
	\Gr{\vhtid{i}{\delta}}
	= \begin{Bmatrix*}[l]
			\; \vamd{t}{j}{i}{\delta}   & \forall j, \\[4mm]
			\; \vad{t}{i}{jk}           & \forall j, k \text{, if } \delta = \Delta,\; \\[4mm]
			\; \vae{t}{i} & 
		\end{Bmatrix*}.
\end{equation}
%
%
\subsubsection{Initialisation constraints.}
%
%
Lacking any contrary information, we assign maximum division distance $\Delta$ to all (selected) segment hypothesis in time-point $t = 1$. This is realised by forcing all other segmentation variables to $0$.
\begin{equation}
	\forall\, 0 \le \delta < \Delta,\; \vht[1] \in \Ht[1]\;:\; \vhtd[1]{\delta} = 0
\end{equation}

%
%
% = = = = = = = = = = = = = = = = = = = = = = = = = = = = = = = = = = = = = = = = = = =
\subsection{Eliminating segmentation variables}
\label{sec:varelim}
% = = = = = = = = = = = = = = = = = = = = = = = = = = = = = = = = = = = = = = = = = = =
%
Considering the costs and constraints defined above it can be seen that segmentation variables are redundant in the formulation of the factor graph.
The continuity equality~\eqref{eq:contexit} provides a definition for each segmentation variable in terms of a sum over a set of assignment variables.
Plugging these definitions into \eqref{eq:treeconstraints}, and replacing \eqref{eq:contenter} and \eqref{eq:contexit} by
\begin{equation}
  \forall t \in \set{2, \dots, T-1},\;
%  \forall t \in (1, T):
  \forall \vht \in \Ht:
    \!\!\!\!\!
    \sum_{\vat[t-1] \in \Gl{\vht}}
    \!\!\!\!\!
      \vat[t-1]
  -
    \!\!\!\!\!
    \sum_{\vat \in \Gr{\vht}}
    \!\!\!\!\!
      \vat = 0
\end{equation}
we can eliminate segmentation variables from the constraints.\footnote{%
One might fear that by replacing \vh by a sum over assignment variables we might lose the restriction that \vh is binary, \ie, that at most one of the
  assignments leaving \vh can be active.
Note, however, that this is now effectively ensured by the tree constraints~\eqref{eq:treeconstraints} (with \vht replaced).}

Similarly, the costs $c_{\vh}$ can be dropped, and added to the cost of each exiting assigment $c_{\va}$, where the constraints guarantee that at most one of these is active.\footnote{%
The costs of segmentation hypotheses \vht[T], which have no exiting assignments, are added to each entering assignment instead.}


\end{document}

% vim::set expandtab tabstop=4 softtabstop=2 shiftwidth=2 ft=tex:
